\chapter{Introduction: `Experimental' Mathematics}
This document was created with the help of a custom class file~\cite{example}. A
{\em \LaTeX{} class file}\index{\LaTeX{} class file@LaTeX class file} is a file,
which holds style information for a particular \LaTeX{} class\footnote{You can
find more about classes at \url{http://www.ctan.org/pkg/clsguide}.}. There are some handy options for citing publications. It is possible to print just the year of some publication:~\citeyear{example}. It is also possible to print the name of the author(s) in the form ``Author et al.'': \citeauthor{example}. Finally, there is also a command to print the full list of authors of a publication: \citet*{example}.
This is an example glossary reference: \GLS{HOL}.\\
 
`Experimental' approaches to mathematics are often naively seen as antithetical to the rigorous and formal nature of the discipline. However, if one examines the daily practice of working mathematicians across various sub-fields and timescales, it becomes clear that the process of mathematical discovery is and always has been intertwined with so called `experimental' approaches.
Conjecturing is a crucial element at the heart of mathematical discovery and the ability to conjecture is often seen as a hallmark of mathematical talent. Interesting, non-trivial conjectures can drive progress for centuries and lead to unintended consequences, often birthing entire new fields of mathematics. Observing patterns in data, extrapolating from simple, well-understood cases are some of the tools used to make conjectures.
The development of computing technology has accelerated almost all aspects of this process and human-machine collaboration has a proven track record of producing new mathematics. There are prominent examples of brute-force computational power leading to both new conjectures (cite {BSD}) and theorems (cite {4-color}). 
\par
The rise of artifical intelligence signifies a new era in mathematical discovery. Machine guided approaches have already lead to new results in number theory, geometry, knot theory, representation theory etc. Crucially, many new results have been found on datasets that have existed for multiple decades and had already been actively studied. However, it is unclear how to judge these discoveries and AI-guided mathematics in general. The recently introduced `Birch Test' \cite{heTriumvirateAIDriven2024} captures some of the essence of
how the mathematical community judges results from AI-guided mathematics.
Recently, LLM approaches have outperformed state of the art reasearch mathematics in certain 
targeted tasks (cite {funsearch}) and combined with rapid developments in proof-assistants, it is clear that machine-guided mathematics is burgeoning (cite {Deepmind stuff, T.Tao, Gowers stuff}).
Humans have notoriously underestimated the rate of progress of AI in impacting their field and I would argue that we are at a tipping point in modern mathematics. The arsenal of a working mathematician should include some familiarity with these tools and importantly we are at the stage where we can influence the growth of the field of machine guided mathematics.

In light of these developments, I aim to address the following questions in my PhD:
\begin{itemize}
    \item What is an `interesting' conjecture? : Despite its impact and importance, the process of conjecturing is mysterious and there is no accepted consensus on what is a `good' versus a `bad' conjecture. 
    I aim to provide a theoretical framework to automate the process of conjecturing and implement this via machine-guided tools. I intend to demonstrate its utility by discovering `interesting' conjectures. The framework will be based on a combination of useful ideas from model theory, functional analysis and machine learning.
    \item Can conjecturing be combined with proof-assistants to accelerate mathematical discovery? : Is there an analogous framework on formal proofs/theorems?
    \item How can these tools be incorporated into the workflow of the practicing mathematician?
\end{itemize}


\section{A historical perspective}

\section{Conjectures as a driving force in mathematics}
\subsection{Overview}
\subsection{Lakatos stuff, good versus bad conjectures}
\section{Machine guided mathematics: A modern perspective}
\subsection{An overview of the literature}
\subsection{Combine conjecturing with machine-guided approaches}
\subsection{How to define interestingness}