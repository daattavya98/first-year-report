\section{Model theory of Banach Spaces}
cite Jose Iovino's stuff. Making notes here so I keep track of them.
To be able to apply model theoretic ideas to the conjecture space, we will need a series of definitions to talk about the relevant objects.\\

Let $[-\infty, \infty]$ denote the set $\R\cup\{-\infty, \infty\}$ with the following topology: The open intervals $(a,b)$ for $a,b\in \R$. The basic neigbourhoods of $-\infty, \infty$ are of the form $[-\infty, a]$ and $(a,\infty]$ respectively for $a\in \R$.
\begin{remark}
    There is a uniform structure compatible with this topology that has a countable base, $\mathfrak{U}$.
\end{remark}

\begin{definition}[Real-valued m-ary relation]
    A real-valued m-ary relation on a normed space $E$ is a function
    \begin{equation*}
        \crlR:E^m\rightarrow [-\infty, \infty]
    \end{equation*}
    which is uniformly continuous on every bounded subset of $E^m$.    
\end{definition}

\begin{definition}[normed space structure]
    A normed space structure is a structure of the form
    \begin{equation*}
        \mathbf{E} = (E, f_i, \crlR_j : i\in I, j\in J),
    \end{equation*}
    where
    \begin{itemize}
        \item $E$ is a normed space over the field of real numbers.
        \item Each $f_i$ if a function $f_i : E^m\rightarrow E$ for some positive integer $m$, and $f_i$ is uniformly continuous on every bounded subset of $E^m$.
        \item Each $\crlR_j$ is a real-valued relation.
    \end{itemize}
\end{definition}

Complete normed space structures are \textit{Banach Space Structures}. One can now define a suitable \textit{language} for normed space structures by
ensuring the following: function and relation symbols for each function and relation in the structure, moduli of uniform continuity for each such symbol on each bounded subset
and bounds for the function symbol on each bounded subset. (skipping technical details for now)

% \begin{definition}[Linear Isometry]
%     Given two normed vector spaces $X$ and $Y$, a linear isometry is a linear map $T:X\rightarrow Y$ if it preserves the norm, i.e. $\|Tx\|_Y = \|x\|_X \: \forall x\in X$.
% \end{definition}

Now, we make an analogous definition of well-formed formulas. Namely, the \textit{postive bounded formulas} on these Banach space structures. Interpretation and the syntax-semantics
relationship is defined in the usual way.

\begin{remark}
    \textbf{APPROXIMATION OF FORMULAS: } There is an interesting notion of approximation of positive bounded formulas, defined by relaxing the norm estimates in the orginal positive bounded formula.
    This defines a partial order and induces a order topology. Interestingly, it seems like this order might be connected to the boundary of the conjecture space? I hypothesize. Also, is there a link between this order topology
    and the general model theory of different kinds of orders? In particular, can I find a link to o-minimality maybe? Super speculative of course.
\end{remark}

Approximations are built inductively in the following way:
\begin{align*}
    & \varphi & & \text{ Approximations of }\varphi\\\\
    & ||t||\leq M & & ||t||\leq N, \text{ where } N > M\\
    & \crlR(t_1,...,t_m)\leq M & & \crlR(t_1,...,t_m)\leq N, \text{ where } N > M\\
    & \exists x(||x||\leq M \wedge \sigma) & & \exists x(||x||\leq N \wedge \sigma^{'}), \text{ where } N > M \text{ and } \sigma^{'} \text{ is an approximation of }\sigma
\end{align*}
with similar rules of disjunction and atomic positive bounded formulas of the opposite type. $\varphi < \varphi^{'}$ if $\varphi^{'}$ is an approximation of $\varphi$ and the relation $ < $ is a partial order in $L$. The sets
\begin{equation*}
    [\phi, \psi) = \{\sigma : \phi \leq \sigma < \psi\}, \quad \phi < \psi
\end{equation*}
form a basis for the order topology on $L$.

\begin{lemma}[Perturbation Lemma]
    For every positive bounded $L$-formula, $\varphi(x_1,...,x_n)$, 
    every $\varphi^{'}>\varphi$ and every $M > 0$, there exists $\delta > 0$ such that:
    if $\mathbf{X}$ is a Banach space $L$-structure and $a_1,...,a_n$ are elements of the universe of $\mathbf{X}$ such that
    \begin{equation*}
        \mathbf{X}\models \bigwedge_{1\leq i\leq n} ||a_i||\leq M \wedge \varphi[a_1,...,a_n],
    \end{equation*}
    then whenever $b_1,...,b_n$ are elements of the universe of $\mathbf{X}$ satisfying
    \begin{equation*}
        \max_{1\leq i\leq n} ||a_i - b_i|| < \delta,
    \end{equation*}
    we have
    \begin{equation*}
        \mathbf{X}\models\varphi^{'}[b_1,...,b_n]
    \end{equation*}
\end{lemma}

\begin{definition}[Approximate Satisfaction]
    $\mathbf{X}$ approximately satisfies $\varphi[a_1,...,a_n]$, written as $\mathbf{X}\models_{\mathcal{A}}\varphi[a_1,...,a_n]$ if for every approximation $\varphi^{'}$ of $\varphi$, $\mathbf{X}\models\varphi^{'}[a_1,...,a_n]$.
\end{definition}

So $\mathbf{X}$ approximately satisfies a set $\Gamma(x_1,...,x_n)$ of positive bounded formulas \textit{iff} $\mathbf{X}\models\Gamma_{+}[a_1,...,a_n]$ where $\Gamma_{+}$ is the set of all approximations of formulas in $\Gamma$.
\begin{remark}
    This notion of approximate satisfaction rather than the usual notion of satisfaction provides the correct semantics for a model-theoretic analysis of Banach space structures.
\end{remark}

Read through the beginning section of Iovino's book related to $\crlU$, ultraproducts, ultrafilters etc. for Banach space structures.
TODO: Include definition of filter and ultrafilter if it feels necessary later.

Don't understand the banach space model theory compactness theorem yet (As in the proof). Moving on to see if this stuff is going to be useful.

Is the stuff on defining a real-valued relation useful?